\documentclass{beamer}
\setbeamertemplate{caption}[numbered]
\usepackage{graphicx}
\usepackage[utf8]{inputenc}
\usepackage[english]{babel}
\graphicspath{{Images/}}
\usepackage{comment}
\usepackage{verbatim}
\usepackage{hyperref}
\mode<presentation>
{
	\usetheme{AnnArbor}
	\usecolortheme{crane}
}

\title[Data Management Using Stata]{Data Management Using Stata}
\subtitle[ISRC Workshop]{Iowa Social Research Center (ISRC) Workshop}
\author[Wallace]{Desmond D. Wallace}
\institute[University of Iowa]{Department of Political Science\\The University of Iowa\\Iowa City, IA}

\date{September 22, 2017}

\begin{document}
	
\begin{frame}
	\titlepage
\end{frame}

\section{Data Input/Output}
\subsection{Importing Data}

\begin{frame}
	\frametitle{RECALL: Opening a Dataset}
		\begin{itemize}
			\item The command for opening a dataset in Stata is \texttt{use}.
			\item If a dataset is already open, opening a new dataset requires including the option \texttt{clear} with the \texttt{use} command.
			\item Examples
				\begin{itemize}
					\item Example: \texttt{use \textit{filename}} works if there is no data in Stata's memory.
					\item Example: \texttt{use \textit{filename}, clear} works if data is already in memory.
				\end{itemize}
		\end{itemize}
\end{frame}

\begin{frame}
	\frametitle{Importing Data}
		\begin{itemize}
			\item Can read non .dta files into memory using the \texttt{import} command.
				\begin{itemize}
					\item Excel files: \texttt{import excel [using] \textit{filename}, firstrow clear}
					\item Delimited files: \texttt{import delimited [using] \textit{filename}, clear}
				\end{itemize}
			\item See \texttt{help import\_{excel}} and \texttt{help import\_{delimited}} for more information.
		\end{itemize}
\end{frame}

\subsection{Exporting Data}

\begin{frame}
	\frametitle{Saving and Exporting Data}
		\begin{itemize}
			\item Use the \texttt{save} and \texttt{saveold} commands to save data in memory to a .dta file.
				\begin{itemize}
					\item Stata 15 and 14: \texttt{save [\textit{filename}], replace}
					\item Stata 13, 12, and 11: \texttt{saveold [\textit{filename}], version(\#) replace}
					\item \textbf{NOTE: Stata 11 through 13 files NOT COMPATIBLE with Stata 14 and 15!!!}
				\end{itemize}
			\item Can export data in memory to Excel and delimited files.
			\begin{itemize}
				\item Excel files: \texttt{export excel [using] \textit{filename}, firstrow(variables) replace}
				\item Delimited files: \texttt{export delimited [using] \textit{filename}, replace}
			\end{itemize}
			\item See \texttt{help export\_{excel}} and \texttt{help export\_{delimited}} for more information.
		\end{itemize}
\end{frame}

\section{Data Manipulation}
\subsection{Sorting Data}

\begin{frame}
	\frametitle{Sorting Data}
		\begin{itemize}
			\item Use the \texttt{sort} and \texttt{gsort} commands to arrange data.
				\begin{itemize}
					\item \texttt{sort} arranges data in ascending order only.
					\item \texttt{gsort [+|-] \textit{varname} [[+|-] \textit{varname} \textellipsis]}
						\begin{itemize}
							\item + -- Sort in ascending order
							\item - -- Sort in descending order
						\end{itemize}
				\end{itemize}
		\end{itemize}
\end{frame}

\subsection{Subsetting Data}

\begin{frame}
	\frametitle{Subsetting Data}
		\begin{itemize}
			\item Use \texttt{drop} or \texttt{keep} in combination with an \texttt{if} or \texttt{in} statement to subset observations.
				\begin{itemize}
					\item \texttt{drop [in \textit{range}] if \textit{exp}} eliminates observations from memory satisfying specified condition(s).
					\item \texttt{keep [in \textit{range}] if \textit{exp}} keeps observations from memory satisfying specified condition(s).
				\end{itemize}
			\item Use \texttt{drop \textit{varlist}} to eliminate variables or \texttt{keep \textit{varlist}} to keep variables
			\item \textbf{NOTE: \texttt{drop} and \texttt{keep} are NOT reversible.}
		\end{itemize}
\end{frame}

\subsection{Creating Variables}

\begin{frame}
	\frametitle{Generating Variables}
		\begin{itemize}
			\item \texttt{\underline{g}enerate} command creates a new variable.
				\begin{itemize}
					\item \texttt{generate [\textit{type}] =\textit{exp} [if] [in]}
					\item If \textit{type} is not specified, variable type is determined by \textit{exp}
				\end{itemize}
			\item \texttt{replace} command replaces the contents of an existing variable.
				\begin{itemize}
					\item \texttt{replace \textit{oldvar} =\textit{exp} [if] [in]}
				\end{itemize}
			\item \texttt{egen} command used to create variables based on special functions.
				\begin{itemize}
					\item \texttt{egen [\textit{type}] \textit{newvar} = \textit{fcn(arguments)} [if] [in], by(\textit{varlist})}
					\item \texttt{by} option is used to calculate values over specified variables.
					\item Functions written for use with \texttt{egen} are ONLY for \texttt{egen}
				\end{itemize}
			\item See \texttt{help generate} and \texttt{help egen} for more information.
		\end{itemize}
\end{frame}

\subsection{Recoding Variables}

\begin{frame}
	\frametitle{Recoding Variables}
		\begin{itemize}
			\item Use the \texttt{recode} command to change values of categorical variables.
				\begin{itemize}
					\item \texttt{recode \textit{varlist} (\textit{rule}) [(\textit{rule}) ...], \textit{\underline{gen}erate(\textit{newvar}) options}}
					\item \texttt{recode \textit{varlist} (\textit{erule}) [(\textit{erule}) ...], \textit{\underline{gen}erate(\textit{newvar}) options}}
				\end{itemize}
			\item Use the \texttt{generate} option to save recoded variable to new variable.
		\end{itemize}
\end{frame}

\begin{frame}
	\frametitle{Recoding Variable Rules}
		\begin{itemize}
			\item \textit{rule}
				\begin{enumerate}
					\item \texttt{\# = \#}
					\item \texttt{\# \# = \#}
					\item \texttt{\#/\# = \#}
					\item \texttt{\underline{nonm}issing = \#}
					\item \texttt{\underline{mis}sing = \#}
				\end{enumerate}
			\item \textit{erule}
				\begin{enumerate}
					\item \texttt{\# | \#/\# = el [``label'']}
					\item \texttt{\underline{nonm}issing = el [``label'']}
					\item \texttt{\underline{mis}sing = el [``label'']}
					\item \texttt{else | * = el [``label'']}
				\end{enumerate}
		\end{itemize}
\end{frame}

\begin{frame}
	\frametitle{Recoding Variable Rules}
		\begin{itemize}
			\item Keywords \texttt{missing}, \texttt{nonmissing}, and \texttt{else} must be the last rules specified.
			\item \texttt{else} cannot be combined with \texttt{missing} or \texttt{nonmissing}.
			\item Must use the \texttt{generate} option when recoding a variable, and specifying value labels.
			\item See \texttt{help recode} for more information.
		\end{itemize}
\end{frame}

\subsection{Summarizing Data}

\begin{frame}
	\frametitle{Summarizing Data}
		\begin{itemize}
			\item Recall, the \texttt{summarize} command is used to report summary statistics for variables.
			\item Can use the \texttt{collapse} command to create a dataset of summary statistics.
				\begin{itemize}
					\item \texttt{collapse [(\textit{stat})] \textit{varlist} [ [(\textit{stat})] \textellipsis~] [if] [in], by(\textit{varlist})}
					\item \texttt{by} option is used to calculate summary statistics over specified variables.
					\item If \textit{stat} is not specified, default statistic calculated is the mean.
					\item \textbf{NOTE: Only the variables specified in \texttt{collapse} command remain in memory.}
					\item See \texttt{help collapse} for more information, including full list of statistics.
				\end{itemize}
		\end{itemize}
\end{frame}

\section{Data Combination}
\subsection{Vertical Combination}

\begin{frame}
	\frametitle{Append Datasets}
		\begin{itemize}
			\item Appending datasets is when one wishes to combine two, or more, datasets vertically.
			\item Adding observations to existing datasets.
			\item \texttt{\underline{ap}pend using \textit{filename} [\textit{filename} ...], [\textit{options}]}
			\item See \texttt{help append} for more information.
		\end{itemize}
\end{frame}

\subsection{Horizontal Combination}

\begin{frame}
	\frametitle{Merge Datasets}
		\begin{itemize}
			\item Merging datasets is when one wishes to combine two, or more, datasets horizontally.
			\item Adding variables to existing datasets.
			\item \texttt{\underline{mer}ge \emph{join} \textit{varlist} using \textit{filename}, [\textit{options}]}
			\item Unlike \texttt{append}, can only use one dataset on file at a time.
			\item Specified variables must uniquely identify observations.
		\end{itemize}
\end{frame}

\begin{frame}
	\frametitle{Example Merging Join Types}
		\begin{itemize}
			\item One-to-one merge (1:1)
				\begin{itemize}
					\item Merging datasets that have the same uniquely identified observations.
					\item Each dataset holds one observation per unique case.
					\item \texttt{\underline{mer}ge 1:1 \textit{varlist} using \textit{filename}, [\textit{options}]}
					\item Does not matter which dataset is master (file loaded into memory) and which dataset is using (file not loaded into memory).
				\end{itemize}
			\item Many-to-one merge (m:1)/One-to-many merge (1:m)
				\begin{itemize}
					\item Merging datasets where one dataset has multiple observations per unit and another dataset has single observation per unit.
					\item \texttt{\underline{mer}ge m:1 \textit{varlist} using \textit{filename}, [\textit{options}]}
					\item \texttt{\underline{mer}ge 1:m \textit{varlist} using \textit{filename}, [\textit{options}]}
					\item Matters which dataset is master and which dataset is using.
				\end{itemize}
		\end{itemize}
\end{frame}

\begin{frame}
	\begin{center}
		\begin{LARGE}
			Any Questions?
		\end{LARGE}
	\end{center}
\end{frame}

\end{document}

\documentclass{beamer}
\setbeamertemplate{caption}[numbered]
\usepackage{graphicx}
\usepackage{svg}
\usepackage[utf8]{inputenc}
\usepackage[english]{babel}
\graphicspath{{Graphs/}}
\usepackage{comment}
\usepackage{verbatim}
\usepackage{hyperref}
\mode<presentation>
{
	\usetheme{AnnArbor}
	\usecolortheme{crane}
}

\title[MLM I]{Multilevel Modeling (Part 1)}
\subtitle[ISRC Workshop]{Basic Random-Intercept and Random-Slope Modeling}
\author[Wallace]{Desmond D. Wallace}
\institute[University of Iowa]{Department of Political Science\\The University of Iowa\\Iowa City, IA}

\date{March 23, 2018}

\begin{document}

\begin{frame}
	\titlepage
\end{frame}

%\begin{frame}
%	\frametitle{Table of Contents}
%	\tableofcontents
%\end{frame}

\section{REVIEW: Regression Basics}

\begin{frame}
	\frametitle{Regression Highlights}
	\begin{itemize}
		\item A way to summarize the relationship between variables.
		\item Assuming there is a relationship between Y and the independent variable(s).
		\item Relationship may be linear (OLS) or non-linear (CLDV).
		\item Regression helps our understanding of how our dependent variable of interest changes when one or more independent variables vary, while holding remaining variables fixed.
		\item PRF: $y_{i}=\beta_{0}+\beta_{1}x_{1}+\cdots+\beta_{k}x_{k}+\varepsilon_{i}$
		\item SRF: $y_{i}=b_{0}+b_{1}x_{1}+\cdots+b_{k}x_{k}+e_{i}$
	\end{itemize}
\end{frame}

\subsection{Regression Assumptions}

\begin{frame}
	\frametitle{Model Specification}
		\begin{enumerate}
			\item Linearity in the parameters
			\item The number of observations $n$ must be greater than the number of parameters to be estimated
			\item The regression model is correctly specified
		\end{enumerate}
\end{frame}

\begin{frame}
	\frametitle{Independent Variable(s)}
		\begin{enumerate}
			\item $X$ values are fixed in repeated sampling
			\item Variability in $X$ values
			\item There is no perfect multicollinearity
		\end{enumerate}
\end{frame}

\begin{frame}
	\frametitle{Error Term}
		\begin{enumerate}
			\item Zero mean value of error $(e_i)$
			\item \textbf{Homoscedasticity or equal variance} of $e_i$
			\item No autocorrelation between the errors
			\item Zero covariance between $e_i$ and $X_i$
		\end{enumerate}
\end{frame}

\begin{frame}
	\frametitle{Additional Assumptions}
		\begin{enumerate}
			\item Errors are normally distributed
			\item \textbf{Errors for any two observations are independent of one another}
		\end{enumerate}
\end{frame}

\section{Multilevel Modeling}

\subsection{Violating OLS Assumptions}

\begin{frame}
	\frametitle{Multi-Stage Sampling}
		\begin{itemize}
			\item OLS assumptions imply utilization of Simple Random Sampling (SRS)
			\item However, due to cost-efficiency, multi-stage sampling approaches may be utilized instead.
			\item Researcher may randomly sample grouping units instead of individuals (cluster sampling)
			\item Examples
				\begin{itemize}
					\item Students nested in schools
					\item Respondents nested in states (countries)
					\item Patients nested in hospitals
				\end{itemize}
		\end{itemize}
\end{frame}

\begin{frame}
	\frametitle{Applying OLS to Multilevel Data}
		\begin{itemize}
			\item Biased standard errors
			\item Model Misspecification
		\end{itemize}
\end{frame}

\subsection{Introduction}

\begin{frame}
	\frametitle{New Approach}
		\begin{itemize}
			\item Best approach to analyzing nested data is a statistical approach that accounts for both within-group and between-group variation \textit{simultaneously}
			\item One approach is to conceive within-group and between-group variation as random variability
			\item One can achieve this by including \textit{random coefficient(s)} in the statistical model
		\end{itemize}
\end{frame}

\begin{frame}
	\frametitle{Multilevel Model}
		\begin{itemize}
			\item Multilevel Model (MLM) is a model where the parameters vary at more than one level
			\item Features more than one error term
			\item Variation can occur wit respect to the intercept (\textit{random intercept}) and/or the slope (\textit{random slope})
			\item This approach leads to corrected standard errors and correct model specification
		\end{itemize}
\end{frame}

\begin{frame}
	\frametitle{Multilevel Model}
		\begin{itemize}
			\item Model coefficients are now a combination of both fixed and random components
				\begin{itemize}
					\item Fixed Coefficient -- An unknown constant of nature
					\item Random Coefficient -- One which varies from sample of groups to sample of groups
				\end{itemize}
			\item Random coefficients are not estimated, and are instead predicted
			\item Instead of BLUE (Best Linear Unbiased Estimator)  coefficients, we know have BLUP (Best Linear Unbiased Prediction) coefficients
		\end{itemize}
\end{frame}

\subsection{Random Intercept Model}

\begin{frame}
	\frametitle{Null Model}
		\begin{itemize}
			\item Predicting the outcome from only an intercept that varies between groups
			\item The null model takes the following form:
				\begin{itemize}
					\item Level-1 Model: $y_{ij}=\beta_{0j}+\varepsilon_{ij}$
					\item Level-2 Model: $\beta_{0j}=\gamma_{00}+U_{0j}$ where
						\begin{itemize}
							\item $\beta_{0j}$ -- Average (general) intercept holding across all groups (fixed effect)
							\item $U_{0j}$ -- Group-specific effect on the intercept (random effect)
						\end{itemize}
					\item Full Specification: $y_{ij}=\gamma_{00}+U_{0j}+\varepsilon_{ij}$
				\end{itemize}
			\item Interested in general mean value for $y_{ij}$ $(\gamma_{00})$ and deviation between overall mean and group-specific effects for the intercept $(U_{0j})$
		\end{itemize}
\end{frame}

\begin{frame}
	\frametitle{Null Model Assumptions}
		\begin{enumerate}
			\item Groups are a random sample from the population of all possible groups
			\item $U_{0j}$ is randomly drawn from a population distribution with mean $0$ and variance $\tau^{2}_{0}$
			\item $\tau^{2}_{0}$ (Variance of $U_{0j}$) and $\sigma^2$ (variance of $\varepsilon_{ij}$) are uncorrelated
		\end{enumerate}
\end{frame}

\begin{frame}
	\frametitle{Variance of $y_{ij}$}
		\begin{itemize}
			\item Variance of  $y_{ij}$ is just the sum of the level-two and level-one variances
				\begin{itemize}
					\item $Var(y_{ij})=Var(U_{0j})+Var(\varepsilon_{ij})=\tau^{2}_{0}+\sigma^2$
				\end{itemize}
			\item Covariance between two observations from the same group $(ij\mbox{ and }i'j)$ is equal to the variance of the contribution $U_{0j}$ that is shared by these observations
				\begin{itemize}
					\item $Cov(y_{ij},y_{i'j})=Var(U_{0j})=\tau^{2}_{0}$
				\end{itemize}
		\end{itemize}
\end{frame}

\begin{frame}
	\frametitle{Intraclass Correlation Coefficient}
		\begin{itemize}
			\item A measure of the proportion of variation in the dependent variable that occurs \textit{between} groups versus the total variation (\textit{between} and \textit{within})
			\item Correlation between two randomly drawn observations from in one randomly drawn group
			\item Ranges between $0$ (no variation between groups) and $1$ (all between-group variation but no within-group variation)
			\item $\rho(y_{ij},y_{i'j})=\frac{\tau^{2}_{0}}{\tau^{2}_{0}+\sigma^2}$
		\end{itemize}
\end{frame}

\begin{frame}
	\begin{center}
		\begin{LARGE}
			Email: desmond-wallace@uiowa.edu\\
			Any Questions?
		\end{LARGE}
	\end{center}
\end{frame}

\end{document}
